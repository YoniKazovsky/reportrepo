\documentclass{article}

\usepackage{tikz} 
\usetikzlibrary{automata, positioning, arrows} 

\usepackage{amsthm}
\usepackage{amsfonts}
\usepackage{amsmath}
\usepackage{amssymb}
\usepackage{fullpage}
\usepackage{color}
\usepackage{parskip}
\usepackage{hyperref}
  \hypersetup{
    colorlinks = true,
    urlcolor = blue,       % color of external links using \href
    linkcolor= blue,       % color of internal links 
    citecolor= blue,       % color of links to bibliography
    filecolor= blue,        % color of file links
    }
    
\usepackage{listings}

\definecolor{dkgreen}{rgb}{0,0.6,0}
\definecolor{gray}{rgb}{0.5,0.5,0.5}
\definecolor{mauve}{rgb}{0.58,0,0.82}

\lstset{frame=tb,
  language=haskell,
  aboveskip=3mm,
  belowskip=3mm,
  showstringspaces=false,
  columns=flexible,
  basicstyle={\small\ttfamily},
  numbers=none,
  numberstyle=\tiny\color{gray},
  keywordstyle=\color{blue},
  commentstyle=\color{dkgreen},
  stringstyle=\color{mauve},
  breaklines=true,
  breakatwhitespace=true,
  tabsize=3
}

\newtheoremstyle{theorem}
  {\topsep}   % ABOVESPACE
  {\topsep}   % BELOWSPACE
  {\itshape\/}  % BODYFONT
  {0pt}       % INDENT (empty value is the same as 0pt)
  {\bfseries} % HEADFONT
  {.}         % HEADPUNCT
  {5pt plus 1pt minus 1pt} % HEADSPACE
  {}          % CUSTOM-HEAD-SPEC
\theoremstyle{theorem} 
   \newtheorem{theorem}{Theorem}[section]
   \newtheorem{corollary}[theorem]{Corollary}
   \newtheorem{lemma}[theorem]{Lemma}
   \newtheorem{proposition}[theorem]{Proposition}
\theoremstyle{definition}
   \newtheorem{definition}[theorem]{Definition}
   \newtheorem{example}[theorem]{Example}
\theoremstyle{remark}    
  \newtheorem{remark}[theorem]{Remark}

\title{CPSC-354 Report}
\author{Yoni Kazovsky  \\ Chapman University}

\date{\today} 

\begin{document}

\maketitle

\begin{abstract}
...
\end{abstract}

\setcounter{tocdepth}{3}
\tableofcontents

\section{Introduction}\label{intro}

(Delete and Replace): This report will document your learning throughout the course. It will be a collection of your notes, homework solutions, and critical reflections on the content of the course. Something in between a semester-long take home exam and your own lecture notes.\footnote{One purpose of giving the report the form of lecture notes is that self-explanation is a technique proven to help with learning, see Chapter 6 of Craig Barton, How I Wish I'd Taught Maths, and references therein. In fact, the report can lead you from self-explanation (which is what you do for the weekly deadline) to explaining to others (which is what you do for the final submission). Another purpose is to help those of you who want to go on to graduate school to develop some basic writing skills. A report that you could proudly add to your application to graduate school (or a job application in industry) would give you full points.}

To modify this template you need to modify the source \texttt{report.tex} which is available in the course repo. For guidance on how to do this read both the source and the pdf of \texttt{latex-example.tex} which is also available in the repo. Also check out the usual resources (Google, Stackoverflow, LLM, etc). It was never as easy as now to learn a new programming lanugage (which, btw, \LaTeX{} is).

For writing \LaTeX{} with VSCode use the \href{https://marketplace.visualstudio.com/items?itemName=James-Yu.latex-workshop}{LaTeX Workshop} extension. 

There will be deadlines during the semester, graded mostly for completeness. That means that you will get the points if you submit in time and are on the right track, independently of whether the solutions are technically correct. You will have the opportunity to revise your work for the final submission of the full report.

The full report is due at the end of the finals week. It will be graded according to the following guidelines.

Grading  guidelines (see also below):
\begin{itemize}
\item Is typesetting and layout professional? 
\item Is the technical content, in particular the homework, correct?
\item Did the student find interesting references~\cite{bla} and cites them throughout the report?
\item Do the notes reflect understanding and critical thinking?
\item Does the report contain material related to but going beyond what we do in class?
\item Are the questions interesting?
\end{itemize}

Do not change the template (fontsize, width of margin, spacing of lines, etc) without asking your first.

\section{Week by Week}\label{homework}

\subsection{Week 1}



\subsubsection*{Notes}

In the reading about the mu puzzle, I learned about formal systems. The idea is that within the rules, you can reach the same desired outcome through different paths (at a given point you may have the option to apply several rules but can only chose one at any given step). In this context, a string which is created by applying one of the rules to the previous string is reffered to as a theorem. 

I also learned about discrete mathematics proofs and look forward to learning about how they will be applied to the developement of a programing language.

\subsubsection*{Homework}

NNG Tutorial World Level 5:

a+(b+0)+(c+0)=a+b+c.

Solution: 

First we use the Lean add zero proof to remove the 0 in b+0.

Then we use the Lean add zero proof to remove the 0 in c+0.

Finally we are left with a+b+c = a+b+c and we can use rfl to confirm our proof with reflexive property.

...

NNG Tutorial World Level 6:

a+(b+0)+(c+0)=a+b+c.

Solution: 

First we use the Lean precision add zero proof (tarteting c) to remove the 0 in c+0.

Then we use the Lean add zero proof to remove the remaining 0 in b+0.

Finally we are left with a+b+c = a+b+c and we can use rfl to confirm our proof with reflexive property.

...

NNG Tutorial World Level 7:

For all natural numbers a, we have succ(a)=a+1.

Solution: 

First we unwravel the one with the Lean rw proof to eliminate the one with a succ0.

Then we use the Lean rw proof with add succ to change n + succ0 into succ(n+0).

Then we use the rw proof to rewrite succ(n+0) into succ(n).

Finally we are left with succ(n) = succ(n) and we can use rfl to confirm our proof with reflexive property.

...

NNG Tutorial World Level 8:

For all natural numbers a, we have succ(a)=a+1.

Solution: 

For this problem I simplified both sides of the equation using the rewrite proof with succesors such as 3 = succ 2 and more.

Eventually I got to this point: succ (succ 0) + succ (succ 0) = succ (succ (succ (succ 0)))

At this point I began using rw add succ to simplify the left side of the equation

Then I used rw add zero to remove a remaining zero and I was left with this: succ (succ (succ (succ 0))) = succ (succ (succ (succ 0)))

At this point I used the rfl to confirm my proof reflexively.

...

In all of the above examples I used the Lean rfl proof which directly corresponds to the mathematical reflexive property which states that:
any number a is equal to itself. In other words a = a or b+c = b+c, etc.


%In case you want to draw automata in Latex, you can use the tikz %package. Here is an example of a simple automaton:
%
%\begin{tikzpicture}[shorten >=1pt,node distance=2cm,on grid,auto] 
%  \node[state] (q_1)   {$q_1$}; 
%  \node[state] (q_2) [above right=of q_1] {$q_2$}; 
%  \node[state] (q_3) [below right=of q_2] {$q_3$}; 
%   \path[->] 
%   (q_1) edge  node {0} (q_2)
%         edge  node [swap] {1} (q_3)
%   (q_2) edge  node  {1} (q_3)
%         edge [loop above] node {0} ()
%   (q_3) edge [loop below] node {0,1} ();
%\end{tikzpicture}
%
%By the way, GPT-4 is quite good at outputting tikz code.

\subsubsection*{Comments and Questions}

This section was a good refresh on some of the discrete mathematics concepts that I had forgotten over break. My question for this week is the following:

How are these mathematical concepts applied to the developement of programming languages?


\subsection{Week 2}


\subsection{\ldots}

\ldots

\section{Lessons from the Assignments}

(Delete and Replace): Write three pages about your individual contributions to the project.

On 3 pages you describe lessons you learned from the project. Be as technical and detailed as possible. Particularly valuable are \emph{interesting} examples where you connect concrete technical details with \emph{interesting} general observations or where the theory discussed in the lectures helped with the design or implementation of the project.

Write this section during the semester. This is approximately a quarter of apage per week and the material should come from the work you do anyway. Just keep your eyes open for interesting lessons.

Make sure that you use \LaTeX{} to structure your writing (eg by using subsections).

\section{Conclusion}\label{conclusion}

(Delete and Replace): (approx 400 words) A critical reflection on the content of the course. Step back from the technical details. How does the course fit into the wider world of software engineering? What did you find most interesting or useful? What improvements would you suggest?

\begin{thebibliography}{99}
\bibitem[BLA]{bla} Author, \href{https://en.wikipedia.org/wiki/LaTeX}{Title}, Publisher, Year.
\end{thebibliography}

\end{document}
